\input{/Users/Juste/Documents/ComplexSystems/CityNetwork/Docs/Headers/WPHeader.tex}


\title{Modeling Carpooling : a Brief Literature Review\bigskip\\
\textit{Working Paper}
}

\author{\noun{Juste Raimbault}}
\date{9th April 2015}


\maketitle

\justify


\begin{abstract}
\centering
First brief insight into rare carpooling literature. Not well organized nor equally developped.
\end{abstract}



\paragraph{Carpooling Encouragement Initiatives}

Statistical analysis of a web-based initiative to encourage carpooling to work~\cite{Abrahamse201245}.

\paragraph{Carpooling within a social organisation}

University initiative for campus carpooling described in~\cite{Bruglieri2011558}. Description of a web application proposing travels to users, using heuristic algorithms to solve the so-called ``carpooling problem'' (matching between potential origin/destination/intermediate stops and drivers/carpoolers).

More interesting, proposition of an agent-based model to test possibility of carpooling at manufacturing plant~\cite{Bellemans20121221}. Paper describes only chanllenges to achieve project, and possible data sources (GPS, Bluetooth).



\paragraph{Multi-agent Modeling and Simulation of Carpooling}

First project of platform to simulate carpooling. \cite{Galland2013860} : description of platform architecture. \cite{galland2014multi} : more details on implementation. Includes agents choices and learning. Detailed schedules. \textit{Very few results}, second paper describes implementation and first results of runs (on 1000 synthetic agents, real geographical context, Flandres), e.g. proportion of agents carpooling.

Other conceptual model in~\cite{Cho2012801}. Includes social distances in co-rider choice - utility function for carpoolers. Formal description of the model of~\cite{Bellemans20121221}. \textit{Not implemented yet (?)}.

Both ongoing work.



\paragraph{Co-routing Problem}

Already evoked, issue of matching not necessarily same O/D. example of heuristic in~\cite{Knapen2012821} (Operational Research problem). More elaborated matching in~\cite{Yan2011512} (matching including profiles of carpoolers). Algorithm giving exact Pareto-solution (in multi-objective context) in~\cite{filcek2014common}.


\paragraph{Statistical Analysis}

Data analysis is relevant for an understanding of spatialized use patterns of a car-pooling system, but may be necessary for a modeling approach, for example in the case of a data-driven model which needs precise parametrization from statistics on real data. The influence of factors at various levels (worksite level, company level, economic sector level) may be desirable to integrate in a generic model, and one thus needs to empirically understand each role. In~\cite{vanoutrive2009carpooling}, a multi-level modeling approach is used on Belgian commuting data, which reveals significant influence of the economic sector, whereas no significant results on the influence of employer-level carpooling promotion (what is interesting to our question) were found.


Results of~\cite{Abrahamse201245} are producted by statistical analysis.



\paragraph{Biological Agent-based Modeling of Collective Action}

A crucial point that should be tackled by our modeling approach is the notion of ``critical mass'', i.e. intuitively the typical number of users giving an optimal performance (in the Pareto sense, depending on indicators defined) for the carpooling systems. One can take an abstract viewpoint and understand it in a game-theoretic formulation, where agents take decision on their behavior to the carpooling system in a game. Analogy with biological systems is immediate through abstraction, what allow the use of advanced modeling techniques developed in the field. An abstract model, proposed in~\cite{10.1371/journal.pcbi.1004101}, includes the notion of quorum signalling in a game-theoretic model with an arbitrary number of agents (that can be bacteria, humans, etc. ). The quorum corresponds to our notion of critical mass, and it is proven that communication between agents and the system (signaling) allows the emergence of a collective action without top-down coordination. It suggests that bottom-up measures may have a powerful effect on overall system behavior and must therefore be considered for integration in the modeling process.













%%%%%%%%%%%%%%%%%%%%
%% Biblio
%%%%%%%%%%%%%%%%%%%%

\bibliographystyle{apalike}
\bibliography{/Users/Juste/Documents/ComplexSystems/CarPooling/Biblio/BibTeX/CarPooling}


\end{document}



